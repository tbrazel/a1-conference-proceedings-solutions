\documentclass[english]{article}
\usepackage{amsmath,amssymb,amsthm}
\usepackage{geometry}
\usepackage{hyperref}
\usepackage{color}
\usepackage{mathtools}
\usepackage{mathrsfs}

\usepackage{tikz}
\usetikzlibrary{shapes.geometric, arrows}
\usetikzlibrary{decorations.pathmorphing}
\usetikzlibrary{cd}
\tikzstyle{var} = [rectangle, minimum width=1cm, minimum height=0.5cm, text centered, draw=black, fill=white!30]
\tikzstyle{output} = [rectangle, text centered, draw=black, fill=gray!20]
\tikzstyle{input} = [rectangle, text centered, draw=black, fill=green!20]
\tikzstyle{arrow} = [thick,->,>=stealth]

\setlength{\parindent}{0pt}

\newcommand{\green}[1]{\textcolor{green}{#1}}
\newcommand{\red}[1]{\textcolor{red}{#1}}
\newcommand{\blue}[1]{\textcolor{blue}{#1}}
\newcommand{\purple}[1]{\textcolor{purple}{#1}}
\newcommand{\gray}[1]{\textcolor{gray}{#1}}

\theoremstyle{definition}
\newtheorem{theorem}{Theorem}[subsection]

\newtheorem{corollary}[theorem]{Corollary}
\newtheorem{definition}[theorem]{Definition}
\newtheorem{digression}[theorem]{Digression}
\newtheorem{example}[theorem]{Example}
\newtheorem{exercise}[theorem]{Exercise}
\newtheorem{lemma}[theorem]{Lemma}
\newtheorem{notation}[theorem]{Notation}
\newtheorem{note}[theorem]{Note}
\newtheorem{proposition}[theorem]{Proposition}
\newtheorem{remark}[theorem]{Remark}
\newtheorem{slogan}[theorem]{Slogan}
\newtheorem{warning}[theorem]{Warning}

\newcommand{\thistheoremname}{}
\newtheorem{genericthm}[theorem]{\thistheoremname}
\newenvironment{customenvironment}[1]
  {\renewcommand{\thistheoremname}{#1}%
   \begin{genericthm}}
  {\end{genericthm}}

\renewcommand{\setminus}{\smallsetminus}
\newcommand{\minus}{\smallsetminus}

\newcommand{\iso}{\simeq}
\newcommand{\ceil}[1]{\left\lceil #1 \right\rceil}
\newcommand{\floor}[1]{\left\lfloor #1 \right\rfloor}
\newcommand{\lrangle}[1]{\langle #1 \rangle}

\newcommand{\Aut}{\text{Aut}}
\newcommand{\B}{\text{B}}
\newcommand{\Bez}{\text{B\'ez}}
\newcommand{\BGL}{\text{BGL}}
\newcommand{\Bl}{\text{Bl}}
\renewcommand{\char}{\text{char}}
\newcommand{\cof}{\text{cof}}
\newcommand{\colim}{\text{colim}}
\newcommand{\Disc}{\text{Disc}}
\newcommand{\EKL}{\text{EKL}}
\newcommand{\End}{\text{End}}
\newcommand{\Fix}{\text{Fix}}
\newcommand{\Frac}{\text{Frac}}
\newcommand{\Frob}{\text{Frob}}
\newcommand{\Fun}{\text{Fun}}
\newcommand{\GL}{\text{GL}}
\newcommand{\GW}{\text{GW}}
\newcommand{\Gr}{\text{Gr}}
\newcommand{\grad}{\text{grad}}
\newcommand{\Hom}{\text{Hom}}
\providecommand{\ind}{\text{ind}}
\newcommand{\Jac}{\text{Jac}}
\newcommand{\Map}{\text{Map}}
\newcommand{\op}{\text{op}}
\newcommand{\Pic}{\text{Pic}}
\newcommand{\Pre}{\text{Pre}}
\newcommand{\Proj}{\text{Proj}\hspace{0.1em}}
\newcommand{\PSL}{\text{PSL}}
\newcommand{\rank}{\text{rank}}
\newcommand{\Rep}{\text{Rep}}
\newcommand{\Res}{\text{Res}}
\newcommand{\s}{\text{s}}
\newcommand{\sgn}{\text{sgn}}
\newcommand{\Sh}{\text{Sh}}
\newcommand{\sig}{\text{sig}}
\newcommand{\Sm}{\textit{Sm}}
\newcommand{\SO}{\text{SO}}
\newcommand{\Sp}{\text{Sp}}
\newcommand{\Spc}{\text{Spc}}
\newcommand{\Spec}{\text{Spec}\hspace{0.1em}}
\newcommand{\Spin}{\text{Spin}}
\providecommand{\spn}{\text{span}}
\newcommand{\sSet}{\textit{sSet}}
\newcommand{\SU}{\text{SU}}
\newcommand{\supp}{\text{supp}}
\newcommand{\Sym}{\text{Sym}}

\newcommand{\Th}{\text{Th}}
\newcommand{\Tr}{\text{Tr}}
\newcommand{\type}{\text{type}}


\newcommand{\A}{\mathbb{A}}
\newcommand{\C}{\mathbb{C}}
\newcommand{\F}{\mathbb{F}}
\newcommand{\Q}{\mathbb{Q}}
\newcommand{\R}{\mathbb{R}}
\newcommand{\Z}{\mathbb{Z}}

\providecommand{\CP}{\mathbb{C}\mathrm{P}}
\renewcommand{\O}{\mathcal{O}}
\renewcommand{\H}{\textbf{H}}

\let\oldemptyset\emptyset
\let\emptyset\varnothing

\let\phi\varphi
\let\bar\overline
\let\tilde\widetilde
\newcommand{\gw}[1]{\left\langle #1 \right\rangle}

\providecommand\smsh{\wedge}




\newcommand{\G}{\mathbb{G}}
\renewcommand{\P}{\mathbb{P}}
\newcommand{\E}{\mathbb{E}}
\newcommand{\und}[1]{\underline{#1}}

\newcommand{\po}{\arrow[ul,phantom,"\ulcorner" very near start]}
\newcommand{\pb}{\arrow[dr,phantom,"\lrcorner" very near start]}
\newcommand{\xto}[1]{\xrightarrow{#1}}
\newcommand{\from}{\leftarrow}
\newcommand{\xfrom}[1]{\overset{#1}{\leftarrow}}

\newcommand{\hookto}{\xhookrightarrow{}}
\newcommand{\xhookto}[1]{\overset{#1}{\hookrightarrow}}

\newcommand{\hookfrom}{\xhookleftarrow{}}
\newcommand{\xhookfrom}[1]{\xhookleftarrow{#1}}

\newcommand{\tto}{\twoheadrightarrow}
\newcommand{\xtto}[1]{\overset{#1}{\twoheadrightarrow}}
\newcommand{\ffrom}{\twoheadleftarrow}
\newcommand{\xffrom}[1]{\overset{#1}{\ffrom}}

\newcommand{\ladjoint}[2]{ #1\rightleftarrows #2 }
\makeatletter
\newcommand{\superimpose}[2]{%
  {\ooalign{$#1\@firstoftwo#2$\cr\hfil$#1\@secondoftwo#2$\hfil\cr}}}
\makeatother
\newcommand{\smallslash}{\mbox{\tiny/}}

\newcommand{\clhook}{\mathrel{\raisebox{0.1em}{$\mathrel{\mathpalette\superimpose{{\hspace{0.1cm}\vspace{0.1em}\smallslash}{\hookrightarrow}}}$}}}
\newcommand{\xclhook}[1]{\overset{#1}{\clhook}}

\newcommand{\clhookfrom}{\mathrel{\raisebox{0.1em}{$\mathrel{\mathpalette\superimpose{{\hspace{0.1cm}\vspace{0.1em}\smallslash}{\hookleftarrow}}}$}}}

\newcommand{\ohook}{\mathrel{\raisebox{0.03em}{$\mathrel{\mathpalette\superimpose{{\hspace{0.1cm}\vspace{0.03em}\mbox{\small$\circ$}}{\hookrightarrow}}}$}}}

\newcommand{\ohookfrom}{\mathrel{\raisebox{0.03em}{$\mathrel{\mathpalette\superimpose{{\hspace{0.1cm}\vspace{0.03em}\mbox{\small$\circ$}}{\hookleftarrow}}}$}}}

\parindent0em
\parskip0.5em

\usepackage{caption}
\usepackage{subcaption}
\usepackage{longtable}
\usepackage{graphicx}% http://ctan.org/pkg/graphicx
\usepackage{yhmath}% http://ctan.org/pkg/yhmath
\usepackage{mathdots}% http://ctan.org/pkg/mathdots
\usepackage{MnSymbol}% http://ctan.org/pkg/mnsymbol
\usepackage{mathtools}
\usepackage{mdframed}
\usepackage{float}


\usetikzlibrary{intersections}
\usepackage[style=alphabetic]{biblatex}
\bibliography{references}

\providecommand{\RP}{\mathbb{R}\textbf{P}}
\providecommand{\sPre}{\text{sPre}}
\providecommand{\smashprod}{\wedge}
\providecommand{\Top}{\text{Top}}
\renewcommand{\top}{\text{top}}
\providecommand{\hocolim}{\text{hocolim}}
\providecommand{\PGL}{\text{PGL}}
\providecommand{\Nis}{\text{Nis}}
\let\del\partial
\begin{document}

\title{Solutions to exercises: An Introduction to $\A^1$-Enumerative Geometry \\ \large Based on lectures by Kirsten Wickelgren delivered at LMS-CMI ``Homotopy Theory and Arithmetic Geometry: Motivic and Diophantine Aspects''}
\author{Thomas Brazelton}
\date{Last compiled: \today}
\maketitle

Please contact me with any questions or comments on these exercises, or submit a pull request at \href{https://github.com/tbrazel/a1-conference-proceedings-solutions}{https://github.com/tbrazel/a1-conference-proceedings-solutions}.



\section*{Solutions}




% Section 1
\setcounter{section}{1}
\setcounter{subsection}{2}

The Grothendieck--Witt ring $\GW(k)$ is generated by elements $\left\langle a \right\rangle$, where $a\in k^\times / \left( k^\times \right)^2$, modulo the following relations:
\begin{enumerate}
\item $\gw{a}\gw{b} = \gw{ab}$
\item $\gw{a} + \gw{b} = \gw{ab(a+b)} + \gw{a+b}$, for $a+b \neq 0$
\item $\gw{a}+\gw{-a} = \gw{1} + \gw{-1}$. We conventionally denote this element as $\H := \gw{1} + \gw{-1}$, called the \textit{hyperbolic element} of $\GW(k)$.
\end{enumerate}
% Exercise 1.2.1
\begin{exercise} In the statements above, (1) and (2) imply (3).
\end{exercise}
\begin{proof} We have that
\begin{align*}
	\gw{a} + \gw{1-a} &= \gw{a(1-a)} + \gw{1}.
\end{align*}
Thus one may see that
\begin{align*}
	\gw{a} + \gw{-a} &= \left( \gw{-a} + \gw{a(1-a)} \right) + \gw{1} - \gw{1-a} \\
	&= \left( \gw{-a^2} + \gw{a^4(1-a)} \right) + \gw{1} - \gw{1-a} \\
	&= \gw{-1} + \gw{1-a} + \gw{1} - \gw{1-a} \\
	&= \H,
\end{align*}
where the second equality follows from noting that $-a + a(1-a) = -a^2$ and $-a \left( a(1-a) \right)(-a^2) = a^4(1-a)$.
\end{proof}




\setcounter{subsection}{3}
\setcounter{theorem}{0}
% Exercise 1.3.1
\begin{exercise}\label{exerc:degree} Compute the $\A^1$-degrees of the following maps:
\begin{enumerate}
\item $\P_k^1 \to \P_k^1$, given by $z \mapsto az$, for $a\in k^\times$.
\item $\P_k^1 \to \P_k^1$, given by $z \mapsto z^2$.
\end{enumerate}
\end{exercise}
\begin{proof} $\ $
\begin{enumerate}
    \item Picking standard affine charts on both copies of $\P^1$, we see that $f$ locally looks like the map $f: \A^1_k \to \A^1_k$ sending $x \mapsto ax$. In particular, it is clear that $f$ has a single isolated root at $0$. Thus
    \begin{align*}
        \deg^{\A^1} f = \deg^{\A^1}_0(f) = \left\langle \Jac(f)(0) \right\rangle = \left\langle a \right\rangle.
    \end{align*}
    
    \item Again picking standard affine charts, we see that $f$ has an isolated root at the origin, however the Jacobian $\Jac(f)(0)$ vanishes. Thus we should consider another target point. Assuming $\char(k) \ne 2$, we can consider the point $1 \in \A^1_k$, for which $f$ has two simple preimages: $f^{-1}(1) = \left\{ \pm 1 \right\}$. We see then that
    \begin{align*}
        \deg^{\A^1} f &= \deg^{\A^1}_1 (f) + \deg^{\A^1}_{-1}(f) = \left\langle \Jac(f)(1) \right\rangle + \left\langle \Jac(f)(-1) \right\rangle \\
        &= \left\langle 2 \right\rangle + \left\langle -2 \right\rangle \\
        &= \H.
    \end{align*}
\end{enumerate}
\end{proof}



% Exercise 1.3.2
\begin{exercise} Compute the B\'ezout bilinear forms of the maps given in Exercise \ref{exerc:degree}.
\end{exercise}
\begin{proof} $\ $
\begin{enumerate}
    \item We let $f(z) = az$ and $g = 1$. Then
    \begin{align*}
        \frac{f(X)g(Y) - f(Y)g(X)}{X-Y} &= \frac{aX - aY}{X-Y} = a.
    \end{align*}
    Thus $\Bez(f/g) = \begin{pmatrix} a \end{pmatrix}$ is the rank one form $\left\langle a \right\rangle \in \GW(k)$.

    \item We let $f(z) = z^2$ and $g=1$. Then
    \begin{align*}
        \frac{f(X)g(Y) - f(Y)g(X)}{X-Y} &= \frac{X^2 - Y^2}{X-Y} = X + Y.
    \end{align*}
    Thus the B\'ezout form is given by the matrix
    \begin{align*}
        \begin{pmatrix} 0  & 1 \\ 1 & 0\end{pmatrix} = \H.
    \end{align*}
    
\end{enumerate}

\end{proof}


\setcounter{subsection}{4}
\setcounter{theorem}{0}
% Exercise 1.4.1
\begin{exercise}\cite{Asok-notes} Prove that naive $\A^1$-homotopy fails to be a transitive relation on hom-sets by considering three morphisms $\Spec k \to \Spec k[x,y]/(xy)$ identifying the points $(0,1)$, $(0,0)$, and $(1,0)$.
\end{exercise}
\begin{proof} We name the three maps $a$, $b$, and $c$, identifying the points $(0,1)$, $(0,0)$, and $(1,0)$, respectively. The affine scheme $\Spec k[x,y]/(xy)$ is the union of the $x$-axis and $y$-axis, as shown below:
\begin{figure}[H]
\begin{center}
\begin{tikzpicture}[scale=0.75]
    \draw[<->] (2,0) -- (-2,0);
    \draw[<->] (0,2) -- (0,-2);


    \coordinate[label =  right:$a$] (a) at (0,1);

    \coordinate[label = below left:$b$] (b) at (0,0);

    \coordinate[label = below:$c$] (c) at (1,0);

    \node at (a)[circle,fill,inner sep=1.5pt]{};
    \node at (b)[circle,fill,inner sep=1.5pt]{};
    \node at (c)[circle,fill,inner sep=1.5pt]{};

\end{tikzpicture}
\end{center}
\caption{$\Spec k[x,y]/(xy)$}%
\end{figure}

This clearly admits a map $\Spec k[y] \to \Spec k[x,y]/(xy)$, which is geometrically the inclusion of the $y$-axis. It is easy to see this hits the point $b$ at time zero, and the point $a$ at time $1$, thus it defines a naive $\A^1$-homotopy $b\sim a$. Similarly, we can include the $x$-axis $\Spec k[y] \to \Spec k[x,y]/(xy)$, which yields a naive $\A^1$-homotopy $b\sim c$.

We claim there is no map $\Spec k[t] \to \Spec k[x,y]/(xy)$ which provides an $\A^1$-homotopy between $a$ and $c$. We first look at the possible ring homomorphisms
\begin{align*}
    k[x,y]/(xy) \to k[t].
\end{align*}
If $x$ maps to something nonzero, then $y$ must map to zero for the relation $xy=0$ to hold, and conversely if $y$ maps to something nonzero then $x$ must map to zero. This implies that any map of schemes $\Spec k[t] \to \Spec k[x,y]/(xy)$ factors through the inclusion of one of the axes, and in particular it cannot hit both $a$ and $c$ in its image.


\end{proof}



\setcounter{subsection}{5}
\setcounter{theorem}{2}
% Exercise 1.5.3
\begin{exercise} Show that the diagram
\[
	\begin{tikzcd}
	X\times Y\rar\dar & X\dar \\
	Y\rar & \Sigma (X\smashprod Y)\po
	\end{tikzcd}
\]
is a homotopy pushout diagram. The context for this example is left ambiguous as the result holds in $\Spc^{\A^1}_{k,\ast}$ just as well as it does for pointed topological spaces.
\end{exercise}
\begin{proof} Consider the diagram:
\[ \begin{tikzcd}
    X\dar & X\vee Y\lar\rar\dar & Y\dar \\
    X\dar & X\times Y\lar\rar\dar & Y\dar \\
    \ast & X\smashprod Y\lar\rar & \ast.
\end{tikzcd} \]
We note that the entries in the bottom row are given by the (homotopy) colimit of the vertical columns above. Since colimits commute, we could take the colimit of each row and then quotient to obtain the colmit down below. We claim that $\colim(X\from X\vee Y \to Y) = \ast$, since all of $Y$ is collapsed to the basepoint in the first map, and all of $X$ is collapsed in the second map.

Then since colimits commute, we get that the column of successive colimits is in fact a colimit:
\[ \begin{tikzcd}
    \ast\dar \\
    \colim(X\from X\times Y \to Y)\dar \\
    \Sigma(X\smashprod Y).
\end{tikzcd} \]
Thus
\begin{align*}
    \colim(X\from X\times Y \to Y)/\ast \cong \colim(X\from X\times Y \to Y) \cong \Sigma(X\smashprod Y).
\end{align*}
\end{proof}



\setcounter{section}{2}
\setcounter{subsection}{1}
\setcounter{theorem}{2}
% Exercise 2.1.3
\begin{exercise}\label{exercise-wekl} $\ $
\begin{enumerate}

	\item Compute the degree of $f : \A_k^2 \to \A_k^2$, given as $f(x,y) = (4x^3, 2y)$ in the case where $\char(k)\neq 2$.
	
	\item Supposing $f$ is \'etale at the origin $0$, show that $w^\EKL(f) = \left\langle \Jac(f)(0) \right\rangle$ is an equality in $\GW(k)$. Show furthermore that an analogous equality holds at any $k$-rational point $x$. 
	
\end{enumerate}
\end{exercise}
\begin{proof} $\ $
\begin{enumerate}
    \item Considering the point $(0,0) \in \A^2_k$, we remark that its preimage is $f^{-1}(0,0) = \left\{ (0,0) \right\}$. We compute the EKL form at this point as follows: let $f = (f_1,f_2)$, and write
    \begin{align*}
        f_1 &= (4x^2)\cdot x + 0\cdot y \\
        f_2 &= 0\cdot x + 2\cdot y.
    \end{align*}
    Then the distinguished socle element is
    \begin{align*}
        E_0(f) := \det \begin{pmatrix} 4x^2 & 0 \\ 0 & 2 \end{pmatrix} = 8x^2.
    \end{align*}
    We remark that the local algebra
    \begin{align*}
        Q_0(f) &= \frac{k[x,y]_{(x,y)}}{(4x^3, 2y)} \cong \frac{k[x]_{(x)}}{x^3}
    \end{align*}
    has a basis as a $k$-vector space given by $\left\{ 1,x,x^2 \right\}$. We can pick $\eta: Q_0(f) \to k$ to be the $k$-linear map determined on basis elements by
    \begin{align*}
        \eta(1) &= 0 \\
        \eta(x) &= 0 \\
        \eta(x^2) &= \frac{1}{8}.
    \end{align*}
    This satisfies $\eta(E_0(f)) = 1$. Consider the Gram matrix $\eta(e_i\cdot e_j)$, for the basis $\left\{ e_1,e_2,e_3 \right\} = \left\{ 1,x,x^2 \right\}$. This is given by
    \begin{align*}
        \begin{pmatrix} \eta(1\cdot 1) & \eta(1\cdot x) & \eta(1\cdot x^2) \\
        \eta(x\cdot 1)) & \eta(x\cdot x) & \eta(x\cdot x^2) \\
        \eta(x^2\cdot 1) & \eta(x^2 \cdot x) & \eta(x^2 \cdot x^2)\end{pmatrix} &= \begin{pmatrix} \eta(1) & \eta(x) & \eta(x^2) \\ \eta(x) & \eta(x^2) & \eta(0) \\
        \eta(x^2) & \eta(0) & \eta(0) \end{pmatrix} \\
        &= \begin{pmatrix} 0 & 0 & \frac{1}{8} \\ 0 & \frac{1}{8} & 0 \\ \frac{1}{8} & 0 & 0 \end{pmatrix}.
    \end{align*}
    As we are allowed to multiply the Gram matrix through by a square without affecting the isomorphism class of the bilinear form in $\GW(k)$, this is equivalent to
    \begin{align*}
        \begin{pmatrix} 0 & 0 & 2 \\ 0 & 2 & 0 \\ 2 & 0 & 0 \end{pmatrix}.
    \end{align*}
    Diagonalizing, we obtain the matrix
    \begin{align*}
        \begin{pmatrix} -2 & 0 & 0 \\ 0 & 2 & 0 \\ 0 & 0 & 2 \end{pmatrix}.
    \end{align*}
    This is exactly $\H + \left\langle 2 \right\rangle$.


    \item Since $f$ is \'etale with an isolated zero at the origin, one sees that the fiber $f^{-1}(0) \to \Spec k$ is \'etale at zero, and hence the local ring $\O_{f^{-1}(0),0}$ must be a separable field extension of $k$. As its residue field is the residue field at the $k$-rational point $0$, one sees that $Q_0(f) \cong \O_{f^{-1}(0),0} \cong k$. Thus the distinguished socle element is equal to the Jacobian $\Jac(f)(0)$, and the local algebra is spanned by the Jacobian. Picking $\eta : Q_0(f) \to k$ sending $\Jac(f)(0)$ to $1$, we see that the EKL form is exactly the rank one form $\left\langle \Jac(f)(0) \right\rangle$. An analogous argument holds for any $k$-rational point $x$.
    
    
    \end{enumerate}

\end{proof}



\setcounter{subsection}{3}
\setcounter{theorem}{2}
% Exercise 2.3.3
\begin{exercise} Compute $\mu^{\A^1}$ for the following ADE singularities over $\Q$:
\begin{center}
	\begin{tabular}{l | l}
	singularity & equation \\
	\hline
	$A_n$	&	$x^2 + y^{n+1}$ \\
	$D_n$	&	$y(x^2 + y^{n-2}) \quad (n\geq 4)$ \\
	$E_6$	&	$x^3 + y^4$ \\
	$E_7$	&	$x ( x^2 + y^3 )$ \\
	$E_8$	&	$x^3 + y^5$.
    \end{tabular}
\end{center}
\end{exercise}
\begin{proof} Recall that $\mu_0^{\A^1}(f) = \deg_0^{\A^1}(\grad f)$. We will compute each of these using the EKL form. We will make frequent use of the following fact: for $u \in k^\times$, the class of the following $n\times n$ anti-diagonal matrix in $\GW(k)$ is given by
\begin{align*}
    \begin{pmatrix} 0 & 0 & \cdots & 0 & u \\
    0 & 0 & \cdots & u & 0 \\
    \vdots & \vdots & \udots & \vdots & \vdots \\
    0 & u & \cdots & 0 & 0 \\
    u & 0 & \cdots & 0 & 0\end{pmatrix} &= \begin{cases} \frac{n}{2}\H & n\text{ is even} \\ \left\langle u \right\rangle + \frac{n-1}{2}\H & n\text{ is odd}. \end{cases}
\end{align*}

\begin{enumerate}
    \item $A_n : x^2 + y^{n+1}$. We see that the gradient is given by $(2x,(n+1)y^n)$. We write this as
    \begin{align*}
        2x &= 2\cdot x + 0\cdot y \\
        (n+1)y^n &= 0\cdot x + (n+1)y^{n-1}\cdot y.
    \end{align*}
    Thus the distinguished socle element is
    \begin{align*}
        E_0(f) &= \det \begin{pmatrix} 2 & 0 \\ 0 & (n+1)y^{n-1} \end{pmatrix} = 2(n+1)y^{n-1}.
    \end{align*}
    A basis for the local algebra
    \begin{align*}
        Q_0(f) &= \frac{k[x,y]_{(x,y)}}{(2x, (n+1)y^n}
    \end{align*}
    is given by $1,y,y^2, \ldots, y^n$. We pick $\eta: Q_0(f) \to k$ given by
    \begin{align*}
        \eta(y^i) &= \begin{cases} \frac{1}{2(n+1)} & i=n \\ 0 & i\ne n \end{cases}.
    \end{align*}
    One verifies that the Gram matrix whose $(i,j)$th entry is $\eta(y^{i-1} y^{j-1})$ is given by
    \begin{align*}
        \frac{1}{2(n+1)}  \begin{pmatrix} 0 & \cdots & 1 \\ \vdots & \udots & \vdots \\ 1 & \cdots & 0 \end{pmatrix} &= \begin{cases} \frac{n}{2}\H & n\text{ is even} \\ \frac{n-1}{2}\H + \left\langle 2(n+1) \right\rangle & n\text{ is odd} \end{cases}
    \end{align*}
    
    \item $D_n: y(x^2 + y^{n-2})$, for $(n\geq 4)$. The gradient is given by $(2yx, x^2 + (n-2)y^{n-3})$. We compute the distinguished socle element by writing
    \[ \left.\begin{aligned}
        2yx &= 2y \cdot x + 0\cdot y \\
        x^2 + (n-2)y^{n-3} &= x\cdot x + (n-2)y^{n-4}\cdot y
    \end{aligned} \right\}  \rightsquigarrow E_0(f) = \det \begin{pmatrix} 2y & 0 \\ x & (n-2)y^{n-4} \end{pmatrix}  = 2(n-2) y^{n-3}.
    \]    
    A basis for $Q_0$ is given by $x, 1, y, y^2, \ldots, y^{n-3}$. Picking $\eta(y^{n-3}) = \frac{1}{2(n-2)}$, and picking $\eta$ to send all other basis elements to zero, we see that the Gram matrix is given by
    \begin{align*}
        \frac{1}{2(n-2)}\begin{pmatrix} -(n-2) & 0 & \cdots & 0 & 0 \\
        0 & 0 & \cdots & 0 & 1 \\
        0 & 0 & \cdots & 1 & 0 \\
        \vdots & \vdots & \udots & \vdots & \vdots \\
        0 & 1 & \cdots & 0 & 0 \end{pmatrix} &= \left\langle -2 \right\rangle + \begin{cases} \frac{n-1}{2}\H & n\text{ is odd}\\ \frac{n-2}{2}\H + \left\langle 2(n-2) \right\rangle & n\text{ is even}. \end{cases}
    \end{align*}
    
    \item $E_6: x^3 + y^4$. The gradient is $(3x^2,4y^3)$, thus one sees
    \begin{align*}
        E_0(f) &= \det \begin{pmatrix} 3x & 0 \\ 0 & 4y^2 \end{pmatrix} = 12xy^2.
    \end{align*}
    Picking the basis $\left\{1,x,y, xy, y^2, xy^2\right\}$ for our local algebra, we pick $\eta(xy^2) = \frac{1}{12}$, and $\eta$ sending all other basis elements to zero. One checks that the EKL class is
    \begin{align*}
        \frac{1}{12} \begin{pmatrix}
        0 & 0 & 0 & 0 & 0 & 1 \\
        0 & 0 & 0 & 0 & 1 & 0 \\
        0 & 0 & 0 & 1 & 0  & 0 \\
        0 & 0 & 1 & 0 & 0  & 0 \\
        0 & 1 & 0 & 0 & 0  & 0 \\
        1 & 0 & 0 & 0 & 0  & 0 \end{pmatrix} &= 3\H.
    \end{align*}
    

    \item $E_7: x(x^2 + y^3)$. The gradient is $(3x^2 + y^3, 3xy^2)$. 
    \[ \left.\begin{aligned}
       3x^2 + y^3 &= (3x)\cdot x + (y^2)\cdot y \\
       3xy^2 &= (3y^2) \cdot x + 0\cdot y
    \end{aligned} \right\}  \rightsquigarrow E_0(f) = \det \begin{pmatrix} 3x & y^2 \\ 3y^2 & 0 \end{pmatrix} = -3y^4.
    \]
    Since $y^3 = -3x^2$ in $Q_0(f)$, we see that $E_0(f) = 9x^2y$ as an element of $Q_0(f)$. We have that $Q_0(f)$ admits a basis given by $\left\{ 1,x,y,x^2,xy,y^2, x^2y \right\}$, and we can pick $\eta(x^2 y) = \frac{1}{9}$, and $\eta$ sending all other basis elements to zero. Then the EKL form is given by
    \begin{align*}
        \frac{1}{9}\begin{pmatrix} 0 & 0 & 0 & 0 & 0 & 0 & 1 \\
        0 & 0 & 0 & 0 & 1 & 0 & 0 \\
        0 & 0 & 0 & 1 & 0 & 0  & 0 \\
        0 & 0 & 1 & 0 & 0 & 0  & 0 \\
        0 & 1 & 0 & 0 & 0 & 0  & 0 \\
        0 & 0 & 0 & 0 & 0 & -3  & 0 \\
        1 & 0 & 0 & 0 & 0 & 0 & 0\end{pmatrix} &= 3\H + \left\langle -3 \right\rangle.
    \end{align*}
    
    \item $E_8: x^3 + y^5$. We see that the gradient is $(3x^2, 5y^4)$, and thus that
    \begin{align*}
        E_0(f) = \det \begin{pmatrix} 3x & 0 \\ 0 & 5y^3\end{pmatrix} = 15xy^3.
    \end{align*}
    Then $Q_0(f)$ admits a $k$-basis $\left\{ 1,x,y,xy,y^2, xy^2, y^3, xy^3 \right\}$, and by picking $\eta(xy^3) = \frac{1}{15}$ and all other basis elements mapping to zero, we have that the EKL form is
    \begin{align*}
        \frac{1}{15} \begin{pmatrix} 0 & 0 & 0 & 0 & 0 & 0 & 0 & 1 \\
        0 & 0 &0 & 0 & 0 & 0 & 1 & 0 \\
        0 & 0 & 0 & 0 & 0 & 1 & 0  & 0 \\
        0 & 0 & 0 & 0 & 1 & 0 & 0  & 0 \\
        0 & 0 & 0 & 1 & 0 & 0 & 0  & 0 \\
        0 & 0 & 1 & 0 & 0 & 0 & 0  & 0 \\
        0 & 1 & 0 & 0 & 0 & 0 & 0  & 0 \\
        0 & 0 & 0 & 0 & 0 & 0 & 0  & 0  \end{pmatrix} = 4\H.
    \end{align*}
    
    
\end{enumerate}

\end{proof}



\printbibliography
\end{document}
